\documentclass{article}
\usepackage[bottom=2cm, right=1.5cm, left=1.5cm, top=2cm]{geometry}
\usepackage{amsmath}
\usepackage{amssymb}
\usepackage{amsthm}
\usepackage{enumitem}
\usepackage{exercise} % Exercises Style
\usepackage{graphicx}
\usepackage{caption}
\usepackage{environ}

% Enable Code
\usepackage{minted}
\let \extra T
\newcommand{\vect}[1]{\boldsymbol{#1}}
\newcommand{\vectrm}[1]{\boldsymbol{\rm #1}}
\DeclareMathOperator{\Tr}{Tr}
\DeclareMathOperator{\Cov}{Cov}
\DeclareMathOperator{\Var}{Var}
\DeclareMathOperator{\E}{E}
\DeclareMathOperator{\diag}{diag}

\usepackage{fancyhdr}
\newenvironment{solution}
{\renewcommand\qedsymbol{$\blacksquare$}
\begin{proof}[Solution]$ $}
  {
\end{proof}}

\title{Solutions to Exam 1 }
\author{Rongfei Jin}
\begin{document}

\pagestyle{fancy}
\fancyhf{}%
\fancyhead[L]{\textbf{ DS5220 \ Exam 1 }}
\fancyhead[R]{\textbf{Rongfei Jin}}
\fancyfoot[C]{\thepage}%
\maketitle

\section*{Problem 1}

\begin{enumerate}[label=(\alph*)]
  \item See code above
  \item See code above
  \item
    \begin{enumerate}[label=(\arabic*)]

      \item Gender
        \begin{enumerate}[label=(\roman*)]
          \item Faster multiplication on 0
          \item
            gender encoded in 0,1 is easiser to interpret because the coefficients represents the effect of being male, or no effect if being female
        \end{enumerate}
      \item Income and travel

        inc25p: 1 if income is greater than 25k, 0 otherwise \\
        inc55p: 1 if income is greater than 55k, 0 otherwise \\
        inc95p: 1 if income is greater than 95k, 0 otherwise \\
        tra025p: 1 if travel is greater than .25h, 0 otherwise \\
        tra400p: 1 if travel is greater than 4h, 0 otherwise \\

        \begin{enumerate}[label=(\roman*)]
          \item Design matrix achieves full rank.
            These transformations solve the collinearity problem since if the model has all the condition except the last one, then the last one is determined, so numerically it is more stable.

          \item
            Easier to interpret as the coefficients are simply the effect of income greater than a certain threshold

        \end{enumerate}
    \end{enumerate}

  \item
    The new model has 7 coefficients (without intercept) and NO coefficients are NA. Gender, income greater than 55k, travel time greater than 0.25h and travel time greater than 4h are significant. \\

    Based on the significant coefficients, we can make the following interpretation.

    \begin{enumerate}[label=(\roman*)]
      \item Males are associated with 0.36 more visit than female
      \item Income greater than 55k is associated with 0.5 less visit than income less than 55k
      \item Travel time greater than 0.25h is associated with 0.6 less visit than travel time less than 0.25h
      \item Travel time greater than 4h is associated with 1.8 less visit than travel time less than 4h

    \end{enumerate}

  \item
    The possion regression has the following probability density function
    \[
      p(y|\eta) = \frac{\eta^y e^{-\eta}}{y!}
    \]
    where $\eta = \exp(\beta_0 + \beta_1 X_1 + \ldots + \beta_p X_p)$ is the mean of the possion distribution. Therefore, we have
    \[
      \E(y|x_1, \ldots, x_p) = \eta = \exp(\beta_0 + \beta_1 x_1 + \ldots + \beta_p x_p)
    \]

    Given female, earning \(\$\)65,000 annually, and living two miles from the park. We have the following data
    \(\text{gen} = 0\), \(\text{inc25p} = 1\) \(\text{inc55p} = 1\), \(\text{inc95p} = 0\),\(\text{tra025p} = 0\), \(\text{tra400p} = 0\).

    Therefore, we have
    \[
      \E(y|x_1, \ldots, x_p) = \exp(\beta_0 + \beta_1 0 + \beta_2 1 + \beta_3 1 + \beta_4 0 + \beta_5 0 + \beta_6 0) = \exp(\beta_0 + \beta_2 + \beta_3) = 34.49
    \]
    Given male, earning \(\$\)65,000 annually, and living two miles from the park. We have the following data

    \(\text{gen} = 1\), \(\text{inc25p} = 1\) \(\text{inc55p} = 1\), \(\text{inc95p} = 0\),\(\text{tra025p} = 0\), \(\text{tra400p} = 0\).

    \[
      \E(y|x_1, \ldots, x_p) = \exp(\beta_0 + \beta_1 1 + \beta_2 1 + \beta_3 1 + \beta_4 0 + \beta_5 0) = \exp(\beta_0 + \beta_1 + \beta_2 + \beta_3) = 49.11
    \]

    \[\frac{E(y|\text{male with given conditions})}{E(y|\text{female with given conditions})} = \frac{49.11}{34.49} \approx 1.42\]

    \[\frac{E(y|\text{female with given conditions})}{E(y|\text{male with given conditions})} = \frac{34.49}{49.11} \approx 0.703\]

\end{enumerate}

\section*{Problem 2}

\begin{enumerate}[label=(\alph*)]
  \item See code above
  \item
    \begin{enumerate}[label=(\roman*)]
      \item the center from both approaches are very close
      \item the bootstrapped CI width is wider than the normal CI, which is expected since the normal CI assumes the distribution is normal, but the bootstrapped CI does not make this assumption

    \end{enumerate}
\end{enumerate}

\section*{Problem 3}
\begin{enumerate}[label=(\alph*)]
  \item

    Since \(Y \sim \text{Poisson}(\lambda)\) we have
    \(P(Y=y;\lambda) = \frac{\lambda^y e^{-\lambda}}{y!}\)

    We compute the Momement Generating Function
    \begin{align*}
      M_Y(t) &= \E(e^{tY}) \\
      &= \sum_{y=0}^{\infty} e^{ty} P(Y=y;\lambda) \\
      &= \sum_{y=0}^{\infty} e^{ty} \frac{\lambda^y e^{-\lambda}}{y!} \\
      &= e^{-\lambda} \sum_{y=0}^{\infty} \frac{(\lambda e^t)^y}{y!} \\
      &= e^{-\lambda} e^{\lambda e^t} & \text{Taylor expansion}\\
      &= e^{\lambda (e^t - 1)}
    \end{align*}
    Now we compute mean by deriving the first moment
    \begin{align*}
      \E(Y) &= M_Y'(0) \\
      &= \frac{d}{dt} e^{\lambda (e^t - 1)} \bigg|_{t=0} \\
      &= \lambda e^{\lambda (e^0 - 1)} \\
      &= \lambda
    \end{align*}

    Now we compute the variance by first deriving the second moment
    \begin{align*}
      \E(Y^2) &= M_Y''(0) \\
      &= \frac{d^2}{dt^2} e^{\lambda (e^t - 1)} \bigg|_{t=0} \\
      &= \lambda e^{\lambda (e^0 - 1)} + \lambda^2 e^{\lambda (e^0 - 1)} \\
      &= \lambda + \lambda^2
    \end{align*}

    Now we compute the variance
    \begin{align*}
      \Var(Y) &= \E(Y^2) - \E(Y)^2 \\
      &= (\lambda + \lambda^2) - \lambda^2 \\
      &= \lambda
    \end{align*}

  \item

    \begin{align*}
      p(y | \lambda) &= \frac{\lambda^y e^{-\lambda}}{y!} \\
      &= e^{\log(\frac{\lambda^y e^{-\lambda}}{y!})} \\
      &= e^{y \log(\lambda) - \lambda - \log(y!)} \\
    \end{align*}

  \item
    since \(\log(\lambda) = \beta_0 + \vect{\beta}\cdot \vect{\rm x} = \beta_0 + \vect \beta^T \vectrm{x}\), we have
    \[
      p(y| \vectrm{x}, \beta_0, \vect{\beta}) = \exp\left\{ [\beta_0 + \vect{\beta}^T \vectrm{x}]y -
        \exp\left\{
          \beta_0 + \vect{\beta}^T \vectrm{x}
        \right\}
      - \log(y!)\right\}
    \]

    \[E(y| \vectrm{x}, \beta_0, \vect{\beta}) = \lambda =
      \exp\left\{
        \beta_0 + \vect{\beta}^T \vectrm{x}
      \right\}
    \]
  \item
    \begin{align*}
      \ell(\beta_0, \vect{\beta}| \vectrm{y}, \vectrm{X}) &= \log \prod_{i=1}^{n} p(y_i| \vectrm{x}_i, \beta_0, \vect{\beta}) \\
      &= \sum_{i=1}^{n} \log p(y_i| \vectrm{x}_i, \beta_0, \vect{\beta}) \\
        &= \sum_{i=1}^{n} \left\{ [\beta_0 + \vect{\beta}^T \vectrm{x}_i]y_i -
            \exp\left\{
            \beta_0 + \vect{\beta}^T \vectrm{x}_i
            \right\}
            - \log(y_i!)\right\} \\
        % &= \vectrm{y}^T (\beta_0 \vectrm{1} + \vectrm{\tilde{X}} \vect{\beta}) - \vectrm{1}^T\exp(\beta_0 \vectrm{1} + \vectrm{\tilde X} \vect{\beta}) - \sum_{i=1}^{n} \log(y_i!)
    \end{align*}
\end{enumerate}

\section*{Problem 4}
\begin{enumerate}[label=(\alph*)]
\item 
\begin{align*}
\frac{\partial}{\partial \beta_0}[ -\ell(\beta_0, \vect{\beta}| \vectrm{y}, \vectrm{X}) ]&= -\sum_{i=1}^{n} y_i + \sum_{i=1}^{n}\exp(\beta_0 + \vect{\beta}^T \vectrm{x}_i) 
\end{align*}

\begin{align*}
\frac{\partial}{\partial \vect{\beta}}[ -\ell(\beta_0, \vect{\beta}| \vectrm{y}, \vectrm{X}) ]&= -\sum_{i=1}^{n} y_i \vectrm{x}_i + \sum_{i=1}^{n}\exp(\beta_0 + \vect{\beta}^T \vectrm{x}_i) \vectrm{x}_i
\end{align*}

\item
\begin{align*}
\frac{\partial^2}{\partial \beta_0^2}[ -\ell(\beta_0, \vect{\beta}| \vectrm{y}, \vectrm{X}) ]&= \sum_{i=1}^{n}\exp(\beta_0 + \vect{\beta}^T \vectrm{x}_i) \\
\end{align*}

\begin{align*}
\frac{\partial^2}{\partial \vect{\beta}^2}[ -\ell(\beta_0, \vect{\beta}| \vectrm{y}, \vectrm{X}) ]&= \sum_{i=1}^{n}\exp(\beta_0 + \vect{\beta}^T \vectrm{x}_i) \vectrm{x}_i \vectrm{x}_i^T
\end{align*}

\begin{align*}
\frac{\partial^2}{\partial \beta_0 \partial \vect{\beta}}[ -\ell(\beta_0, \vect{\beta}| \vectrm{y}, \vectrm{X}) ]&= \sum_{i=1}^{n}\exp(\beta_0 + \vect{\beta}^T \vectrm{x}_i) \vectrm{x}_i
\end{align*}

\begin{align*}
\mathrm{H} = \begin{bmatrix}
    \frac{\partial}{\partial \beta_0^2} & \frac{\partial}{\partial \beta_0 \partial \vect{\beta}} \\
    \frac{\partial}{\partial \beta_0 \partial \vect{\beta}} & \frac{\partial}{\partial \vect{\beta}^2}
\end{bmatrix}
\end{align*}

where \(\frac{\partial}{\partial \vect{\beta}^2}\) is a matrix with \(\frac{\partial}{\partial \beta_i \partial \beta_j}\) where\(i = 1,\ldots, p, j = 1,\ldots p\)

\item for the function to be convex, the Hessian matrix must be positive definite.

\item for the ease of computation, we will express the gradient and Hessian in matrix form and let \(\vect \beta' = \begin{bmatrix}
    \beta_0 & \beta_1 & \ldots & \beta_p
\end{bmatrix}^T\),

\[\frac{\partial}{\partial \vect \beta'} = -\vectrm X^T \vectrm y
+ \vectrm X^T \exp(\vectrm X \vect \beta')
\]

\[\frac{\partial^2}{\partial \vect \beta' \partial \vect \beta'^T} = \vectrm X^T \diag(\exp \{\vectrm X \vect \beta'\}) \vectrm X\]
\end{enumerate}

\section*{Problem 5}
To get the loss function, We first remove the constant term \(\log(y!)\) from negative log-likelihood since it does not affect the optimization problem.
\begin{enumerate}[label=(\alph*)]
  \item 
  \begin{align*}
       \mathcal L(\beta_0 \vect \beta | \vectrm y, \vectrm X) &= \sum_{i=1}^{n} \left\{ [\beta_0 + \vect{\beta}^T \vectrm{x}_i]y_i -
            \exp\left\{\beta_0 + \vect{\beta}^T \vectrm{x}_i\right\}
            \right\} \\
  \end{align*}
\end{enumerate}
Then we add the L2 regularization term to the loss function
  \begin{align*}
       \mathcal L_\lambda(\beta_0, \vect \beta | \vectrm y, \vectrm X) &= \sum_{i=1}^{n} \left\{ [\beta_0 + \vect{\beta}^T \vectrm{x}_i]y_i -
            \exp\left\{\beta_0 + \vect{\beta}^T \vectrm{x}_i\right\}
            \right\} + \lambda ||\vect \beta||_2^2
  \end{align*}

  We then derive the gradient and Hessian of the loss function

  \begin{align*}
    \frac{\partial}{\partial \beta_0} \mathcal L_\lambda(\beta_0, \vect \beta | \vectrm y, \vectrm X) &= -\sum_{i=1}^{n} y_i + \sum_{i=1}^{n}\exp(\beta_0 + \vect{\beta}^T \vectrm{x}_i) \\
    \frac{\partial}{\partial \vect{\beta}} \mathcal L_\lambda(\beta_0, \vect \beta | \vectrm y, \vectrm X) &= -\sum_{i=1}^{n} y_i \vectrm{x}_i + \sum_{i=1}^{n}\exp(\beta_0 + \vect{\beta}^T \vectrm{x}_i) \vectrm{x}_i - 2\lambda\vect \beta \\
    \frac{\partial^2}{\partial \beta_0^2} \mathcal L_\lambda(\beta_0, \vect \beta | \vectrm y, \vectrm X) &= \sum_{i=1}^{n}\exp(\beta_0 + \vect{\beta}^T \vectrm{x}_i) \\
    \frac{\partial^2}{\partial \vect{\beta}^2} \mathcal L_\lambda(\beta_0, \vect \beta | \vectrm y, \vectrm X) &= \sum_{i=1}^{n}\exp(\beta_0 + \vect{\beta}^T \vectrm{x}_i) \vectrm{x}_i \vectrm{x}_i^T - 2\lambda \vectrm I \\
    \frac{\partial^2}{\partial \beta_0 \partial \vect{\beta}} \mathcal L_\lambda(\beta_0, \vect \beta | \vectrm y, \vectrm X) &= \sum_{i=1}^{n}\exp(\beta_0 + \vect{\beta}^T \vectrm{x}_i) \vectrm{x}_i \\
  \end{align*}

  we can write the gradient and Hessian in matrix form as
  \begin{align*}
    \frac{\partial}{\partial \vect \beta'} \mathcal L_\lambda(\beta_0, \vect \beta, \vect \beta' | \vectrm y, \vectrm X) &= -\vectrm X^T \vectrm y
+ \vectrm X^T \exp(\vectrm X \vect \beta') - 2\lambda [0;\vect \beta] \\
    \frac{\partial^2}{\partial \vect \beta' \partial \vect \beta'^T} \mathcal L_\lambda(\beta_0, \vect \beta \vect \beta'| \vectrm y, \vectrm X) &= \vectrm X^T \diag(\exp \{\vectrm X \vect \beta'\}) \vectrm X - 2\lambda (\vectrm I - e_1e_1^T)
  \end{align*}

  It is required that the Hessian matrix is positive definite for the loss function to be convex.




\end{document}
