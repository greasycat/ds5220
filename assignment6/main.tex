\documentclass{article}
\usepackage[bottom=2cm, right=1.5cm, left=1.5cm, top=2cm]{geometry}
\usepackage{amsmath}
\usepackage{amssymb}
\usepackage{amsthm}
\usepackage{enumitem}
\usepackage{exercise} % Exercises Style
\usepackage{graphicx}
\usepackage{caption}
\usepackage{environ}

% Enable Code
\usepackage{minted}
\let \extra T

\newcommand{\vect}[1]{\boldsymbol{#1}}
\DeclareMathOperator{\Tr}{Tr}
\DeclareMathOperator{\Cov}{Cov}
\DeclareMathOperator{\Var}{Var}
\DeclareMathOperator{\E}{E}

\usepackage{fancyhdr}
\newenvironment{solution}
{\renewcommand\qedsymbol{$\blacksquare$}
\begin{proof}[Solution]$ $}{
\end{proof}}

\title{Solutions to Assignment }
\author{Rongfei Jin}
\begin{document}

\pagestyle{fancy}
\fancyhf{}%
\fancyhead[L]{\textbf{ DS5220 \ Assignment 6}}
\fancyhead[R]{\textbf{Rongfei Jin}}
\fancyfoot[C]{\thepage}%
\maketitle
\newpage

\section{Chapter 5 - Conceptual 3}

\begin{enumerate}[label=(\alph*)]


\item
K-fold CV is implemented the following way:
\begin{enumerate}
    \item Split the data into $K$ folds.
    \item For each fold $i$:
    \begin{enumerate}
        \item Use the other $K-1$ folds as training data.
        \item Train the model on the training data.
        \item Evaluate the model on the $i$-th fold.
    \end{enumerate}
    \item Average the evaluation results over all $K$ folds.
\end{enumerate}

\item Compare to the validation set approach, K-fold CV ensures that all data points are used for both training and validation, which can lead to a more reliable estimate of the model's performance. It also reduces the variance of the performance estimate by averaging over multiple folds. 

However, K-fold CV can be computationally more expensive, especially for large datasets or complex models, as it requires training the model $K$ times.

\item  LOOCV is a special case of K-fold CV where $K$ is equal to the number of data points. The K-fold CV where k is less than N is less computationally expensive, as it only requires training the model $K$ times instead of $N$ times. More over, since almost all points are used for training, the variance of error will be greater than that of the K-fold CV, making it more prone to overfitting.  However, if a dataset is small, LOOCV can provide a more accurate estimate of the model's performance, as it uses all but one data point for training.
\end{enumerate}

\section{Chapter 5 - Conceptual 4}

\begin{enumerate}[label=(\alph*)]
\item Repeated sample the original with replacement to create B dataset
\item Train the model on each dataset
\item Predict the Y with the particular X value
\item Compute the standard deviation of the predictions
\end{enumerate}

\section{Chapter 5 - Applied 5}





\end{document}